\documentclass{article}
\usepackage{graphicx} % Required for inserting images
\usepackage{hyperref}
\usepackage{amsmath}
\usepackage[margin=1in]{geometry} % Set margin size to 1 inch
\usepackage{minted}

\title{z-score-normalization using SIMD}
\author{kevin Lopez Chavez}
\date{October 2024}

\newcommand{\zonenorm}{\textit{$Z$-score normalization}}

\begin{document}

\maketitle

% \textbf{Required sections: }
\section{Report deliverables}
\begin{itemize}
    \item The file is named \texttt{report.pdf}.
    \item Links to exact locations where the original program is referenced in \hyperref[sec:links]{Section \ref*{sec:links}}.
    \item An explanation of the code, the program's functionality, flow, and SIMD acceleration in \hyperref[sec:code]{Section \ref*{sec:code}}.
    \item Compilation steps and flags in \hyperref[sec:compilation]{Section \ref*{sec:compilation}}.
    \item Proof of achieved speedup with expectations in \hyperref[sec:speedup]{Section \ref*{sec:speedup}}.
\end{itemize}

% (  5pt) Name the file  "report.pdf" 
% (10pt) Include the link to exact location where the original program is referenced from
% (15pt) Explain your code, functionality of the program, program flow, and how you have used SIMD to accelerate
% (10pt) Compilation steps and flags
% (10pt) Proof of Achieved Speedup ( and if it matches with your exceptions)


\section{Introduction}
The \zonenorm{} is a normalization method used in machine learning to improve the training of models.
While the \zonenorm{} improves the run time of machine learning models, it is costly to compute.
First, one needs to get the mean, then the standard deviation, and then update every input from the matrix.

In this homework, I have made a program that uses SIMD instructions to accelerate the calculation of \zonenorm{} by 5.1x without any drawbacks; the only necessary thing is to execute an x86 processor that supports AVX2 instructions.

% \subsection{ Formula }

% \begin{itemize}
%     \item $\vec{X}$ is the original input.
%     \item $\mu$ is the mean of $\vec{X}$.
%     \item $\sigma$ is the standard deviation of $\vec{X}$.
% \end{itemize}

% (10pt) Include the link to exact location where the original program is referenced from
\section{Links to Exact Locations} \label{sec:links}
I was not able to find exact code for Z normalization, but refer to the following articles to understand the implementation.

\textbf{Explanation and example usage of data normalization}:
\begin{itemize} \item \url{https://www.turing.com/kb/data-normalization-with-python-scikit-learn-tips-tricks-for-data-science} \end{itemize}

\textbf{The code implementation of the Z-score normalization (Scikit-learn repository):}
\begin{itemize}  \item \url{https://github.com/scikit-learn/scikit-learn/blob/main/sklearn/preprocessing/_data.py} \end{itemize}

% (15pt) Explain your code, functionality of the program, program flow, and how you have used SIMD to accelerate
\section{Code Explanation} \label{sec:code}
\input{code_explanation}












% (10pt) Compilation steps and flags
\section{Compilation Steps and Flags} \label{sec:compilation}
I used the following commands to compile the code and run it.
\begin{itemize}
    \item \textbf{Original version:}
          \begin{itemize}
              \item \begin{verbatim} $ g++ ./z_score_norm__Original__.cpp -o z_norm_og.exe && ./z_norm_og.exe \end{verbatim}
          \end{itemize}
    \item \textbf{AVX-optimized version:}
          \begin{itemize}
              \item \begin{verbatim} $ g++ -mavx2 ./z_score_norm__Optimized__.cpp -o z_opt.exe && ./z_opt.exe \end{verbatim}
          \end{itemize}

\end{itemize}
Where the flags mean the following:
\begin{itemize}
    \item \texttt{g++}: Compiler used to compile the C++ code.
    \item \texttt{-mavx2}: This flag enables AVX2 instructions.
    \item \texttt{./z\_score\_norm\_\_Original\_\_.cpp}: The source file.
    \item \texttt{-o z\_norm\_og.exe}: The \texttt{-o}: name of the output executable file.
    \item \texttt{\&\& ./z\_norm\_og.exe}: Execute the program after compiling.
\end{itemize}

The mean should be 0, and the standard deviation should be 1 when a matrix has been normalized.
I performed this operation to validate that the normalization had been done correctly.
This is shown in the output log files \ref{sec: original logs}, \ref{sec: SIMD logs}

% (10pt) Proof of Achieved Speedup ( and if it matches with your exceptions)
\section{Proof of Achieved Speedup} \label{sec:speedup}
Proof of the actual run is shown in Figure~\ref{fig: proof}. Also, I posted the logs (run the programs 15 times) for the original version in section~\ref{sec: original logs} and the SIMD version in section~\ref{sec: SIMD logs}.

\begin{figure}[!h]
    \centering
    \includegraphics[width=1\linewidth]{code_run.png}
    \label{fig: proof}
\end{figure}
The programs were executed 15 times, and the average values are shown below:
\begin{table}[h!]
    \centering
    \begin{tabular}{|c|c|}
        \hline
        \textbf{Version} & \textbf{Average Time (seconds)} \\ \hline
        Original         & 23.10                           \\ \hline
        AVX-Optimized    & 4.47                            \\ \hline
    \end{tabular}
    % \caption{Comparison of average computation times between the original and AVX-optimized versions.}
\end{table}
% \section{Performance Analysis}
% Calculating the ratio of the original execution time to the AVX-optimized execution time shows the speed up. 
\[
    \text{Speed-up} = \frac{\text{Time for Original Version}}{\text{Time for AVX Version}} = \frac{23.10}{4.47} \approx 5.16
\]

Getting a speedup of 5.1 is good, but it is expected to get a speed close to 8(without taking into consideration the additional latency of AVX instructions and data movement)  since we're handling eight data points at one time.
SIMD instructions have a longer latency than regular x86 instructions and require additional instructions to move memory from regular x86 registers to AVX registers, so a 5.1x speed-up is expected.

% This shows that SIMD can achieve significant speed up in \zonenorm{} algorithm. 


\clearpage
\section{Logs for Original File Execution} \label{sec: original logs}
\input{og_logs}

\clearpage
\section{Logs for SIMD File Execution} \label{sec: SIMD logs}
\input{simd_logs}

\end{document}